\documentclass{article}
\usepackage{amsmath, amssymb}
\usepackage{geometry}
\geometry{a4paper, margin=1in}

\begin{document}

\title{Acknowledgements}
\author{Nirupam Khanal / Niv}
\date{06-May-2025}

\maketitle

I want to note that this paper marks the end of my undergraduate education at the University of Southern Mississippi at Hattiesburg, Mississippi. Grandad, an orphan since the age of 6 (if I recall correctly), who grew up in poverty and destitution, saw a dream once where all his kids received the education, healthcare, and shelter essential to living a dignified life. He made his way to Baneras, India where he completed his high school matriculation, before completing his undergraduate education. Then, he completed graduate school. He was one of only 9 ministrants, or \textit{Panchas}, to have attended graduate school when the then King Mahendra Bir Bikram Shah appointed him in the role. He will be glad that not only has everyone in my family attended college (yes, every last relative at least has an undergraduate education), there are over 200 Nepalis at USM alone. \\ \\
I am an international student studying mathematics and physics at the University of Southern Mississippi. Currently, I am engaged in research simulating fluid dynamics using a collocation schema known as Local Method for Approximate Particular Solutions (LMAPS), under the supervision of \textbf{Dr. Huiquing Zhu}. Previously, I have also worked on investigating brachistochrone curve solutions in hyperbolic space using the Poincare and flat-chart metrics, a set of differential equations describing the geodesics in these respective metric spaces, under the supervision of \textbf{Dr. Sungwook Lee}. I hope to attend graduate school and eventually enter academia in both teaching and research capacities. I aspire to work in quantum information theory, differential geometry and topology, and category theory. My ambition is to aid in bridging the aforementioned gap between the world around us and our ability to fluently articulate explanations. In my leisure, I enjoy trying to understand concepts beyond the scope of mathematics as well, with interests in metaphysics, deontology, psychoanalysis, sociology, and literature. I also like to explore my creative interests in poetry, art, and adventure sports. \\ \\ 
The Navier Stokes equations are a couplet of parital differential equations that describe the motion of fluids. No solutions have correctly been identified for the infamous Navier Stokes equations – a set of partial differential equations that represent Newton’s second law of motion for fluids – yet the air and clouds continue to be. I believe there is a fundamental gap in our mathematical paradigms, rather than just gaps in our understanding of phenomenon studied per those paradigms. Well, why does it matter whether we understand the mechanics behind the motion of air and its surroundings? Perhaps you were an extremely pollen-sensitive asthmatic kid like me. As an adult, you are starting to realize how important a comprehensive understanding of evolving pollen maps based on convection patterns is for your safety lately. Dr. Huiqing Zhu initially posed the problem to me, although I been scribbling mathematical gibberish for years. He supervised Gunjan Sah and I for the past two semesters as we tried to incorporate a hermitian interpolation scheme into a localized method of approximate particular solutions. The final revised code on file was provided by him. Additionally, I also want to note \textbf{Dr. C.S. Chen}'s courses on partial differential equations grately helped in my understanding of the system of equations presented. He is also a well-regarded mathematician, known especially for his proficiency in numerical analysis.\\ \\ 
I want to acknowledge some other people who have been essential in my undergraduate journey at USM. 
\begin{enumerate}
    \item Physics advisor: Dr. Jeremy S. Scott (201-2, 361-2) is a theoretical physicist who currently is a Teaching Professor at USM. Famous for both being an academic decent of Werner Heisenberg and for having a laid back outlook to life, I aspire to be as cool as he is despite the immense weight of his expertise in his field. 
    \item Physics Undergraduate mentor: Dr. Michael Vera is the parental figure I have in the physics department as a physics major. He is a superstar physicist and a wonderful educator. 
    \item American Physical Society National Mentoring Community mentor: Dr. Eduardo Mucciolo (UCF) and the American Physical Society (APS) have helped me in many ways since I have served as an Undergraduate Student Ambassador at the University of Southern Mississippi. They organized my attendance of the Global Physics Summit 2025 in Anaheim, CA this past March. 
    \item Lab TA, GA: Ms. Isabella Kirsely (PHY Labs) and Mr. Sidney Gautrau (PHY 327/L) have been the most kind and supportive graduate students and laboratory supervisors anyone could ask for. Their respective fields of radiation physics and applied graph theory, or circuitry, are also very fascinating to me. 
    \item Mathematics advisor: Dr. Sungwook Lee is my advisor in the department of mathematics. He is a parental figure, a mentor, a well-wisher, and a terribly kind human being. He has helped me in every step of my academic journey, including answering some questions I had during the organization and construction of the solutions in this paper. In my mind, his only flaw is his lack of enthusiasm to admit his brilliance with grace. As a token of my immense gratitude for everything he has done for all his students, I would like to leave him with words my father always says to me: \textit{``The child is the father of a man."} I hope to make my parental figures proud, and I will miss Dr. Lee's guidance, affection, and trust when I depart. While I expressed my wishes to put pen to paper with him, he refused and gave me his blessings to depart. I still look forward to a collaboration one day. If I trust my mathematics (which I do), and while I may solve the remaining 5 Millennial prize problems in due time (I am coming for every mathematics tenure-ship on the planet haha; just kidding, I'm going to Bali), I think Dr. Lee would agree that we are first and foremost both enthusiasts for knowledge and dialogue, held down only by gravity. 
    \item KME Honor Society mentor: Dr. Zhifu Xie (MAT 280) is a professor of mathematics and a mentor, and also oversees the activities of the Kappa Mu Epsilon Honor Society at USM. 
    \item Mathematics Graduate mentor: Dr. James V. Lambers (MAT 169) is a professor of mathematics and a mentor. His enthusiasm for student success is known, noted, and appreciated by all the mathematics majors at USM. 
    \item Further acknowledgements: Dr. Jiu Ding (MAT 421/521), Dr. Karen Kohl (MAT 402), Dr. C.S. Chen (MAT 340, 417/517), Dr. John Harris (MAT 326), Dr. Katja Biswas (PHY 350-1), Dr. Parthapratim Biswas, Dr. Khin Maung Maung, Dr. Jessica Valles (SOC 492), Dr. Karen Kozlowski (SOC 101), Dr. Timothy Gutmann (REL 130), Mr. Gaven Wallace (ENG 101), Dr. Katherine Bounds (ENG 102), Mr. Kevin Theil (THE 100), Dr. Joel Webb (HIS 101-2), Dr. Huiqing Zhu (MAT 285, 320, 492 TBD). 
    \item USM affiliations: TRIO-SSS, Hillcrest Hall Residence, The Nations, Nepali Students Society, Muslim Students Association, Carribean and African Students Association, Latin Students Association, Society of Physics Students, Office of Student Affairs, Office of the Provost, School of Mathematics and Natural Sciences 
    \item Non-USM affiliations: House of Khanals, House of Dahals, St. Xavier's School, Rato Bangala School, Youth-led SRHR Advocacy Nepal, Center for Mental Health and Counseling Services Nepal, Himal Association, Asian Schools Debating Championship, World Schools Debating Championship.
\end{enumerate}
While the title of the paper presented may seem facetious and almost childish, it had been my dream to solve a Millennial Prize Problem since a young age. I was 14 the first time I took an attempt at solving the famous Reimann Hypothesis, from number theory. At the time, I barely had any mathematical or logical training. I started to formalize the attached paper on January 1, 2025, at Slidell, Louisiana, during a spontaneous backpacking trip across the state which started in the days prior and would last a few more days. Throughout the month, I reached out to many of you regarding topics and concepts that, at the time, may have given me small hints of inspirations in understanding various components of the Navier Stokes equations and known analytical solutions. I have been at a crossroads this semester regarding my educational and personal future. \\ \\ 
At the end of my time in the South, I continue to hold the inner curious child as my primary identity. I hope everyone is provided with the opportunity and platform to continue engaging with the educational and healthcare apparatus around us. All it took was a simple brown kid sitting on a bench staring at the clouds to figure out that the water continues to be even after the rainfall. 

\end{document}