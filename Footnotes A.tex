\documentclass{article}
\usepackage{amsmath, amssymb, amsthm}
\usepackage{geometry}
\geometry{a4paper, margin=1in}

\title{Quaternionic Transformations and Solutions for PDEs}
\author{Nirupam Khanal / Niv}
\date{26-Feb-2025}

\begin{document}

\maketitle

\tableofcontents 


\section{Context-Invariant Prompts}

\subsection{1. Transformation of PDEs to Quaternionic Frameworks}
\begin{quote}
Define a transformation procedure that systematically maps partial differential equations (PDEs) on real manifolds to algebraic or geometric structures on quaternionic manifolds. Specify how differential operators (e.g., Laplacians) are generalized to quaternionic operators, and describe the constraints/compatibilities required for such a mapping.
\end{quote}

\subsection{2. General Solutions in Quaternionic Settings}
\begin{quote}
Derive the general solution structure for PDEs defined on quaternionic manifolds. Address the role of non-commutative algebra in solution methods (e.g., eigenfunction expansions, integral transforms, or heat kernels), and explain how initial/boundary conditions are incorporated in this framework.
\end{quote}

\section{Key Features of the Prompts}
\begin{itemize}
    \item \textbf{Context-Invariance}: No reliance on specific equations (e.g., heat equation) or domains (e.g., \(\mathbb{R}^n\)).
    \item \textbf{Generality}: Applicable to broad classes of PDEs and quaternionic manifolds.
    \item \textbf{Procedural Focus}: Emphasize \textit{how} transformations/solutions are constructed, not just final results.
\end{itemize}

\subsection{Heat Equation and Transformation to Quaternionic Domain}

\begin{align}
\frac{\partial u}{\partial t} = \alpha^{2} \frac{\partial^{2} u}{\partial x^{2}}\quad ; \quad x\in \mathbb{R} \subseteq \mathbb{C}^{\infty} [0,\infty). 
\end{align} \\
Let \( q \in \mathcal{D} \subset \mathbb{H} \) be a quaternionic variable, where \( \mathbb{H} \) is the algebra of quaternions (with basis \( \{1, i, j, k\} \)), and \( \mathcal{D} \) is a smooth, simply connected domain. We redefine the function \( u(x,t) \) as \( u(q,t) \), now dependent on \( q \) instead of \( x \).

\[
\text{Laplacian:} \quad \nabla_{\mathbb{H}}^2 u = \frac{\partial^2 u}{\partial q_1^2} + \frac{\partial^2 u}{\partial q_2^2} + \frac{\partial^2 u}{\partial q_3^2}.
\]

\[
\text{Heat equation on a Hypercomplex manifold:} \quad \frac{\partial u}{\partial t} = \alpha^2 \nabla_{\mathbb{H}}^2 u \quad ; \quad u : \mathbb{H} \mapsto \mathbb{C}^{\infty}
\]

\subsection{General Solution to the Quaternionic Heat Equation}

\[
\text{Quaternionic heat equation:} \quad \frac{\partial u}{\partial t} - \alpha^2 \overline{\mathcal{F}} \mathcal{F} [u] = 0,
\]
where \(\overline{\mathcal{F}} \mathcal{F} = \nabla_{\mathbb{H}}^2\), is constructed as follows:

\[
\text{Eigenfunction Expansion: }\quad  u(q,t) = \int_{\mathbb{R}} C(\lambda) e^{-\alpha^2 \lambda^2 t} \psi_\lambda(q) \, d\lambda.
\]

\[
\text{Quaternionic Fourier Transform: } \quad \hat{u}(s,t) = \int_{\mathbb{H}} e^{-s \cdot q} u(q,t) \, dq,
\]

\[
\therefore \hat{u}(s,t) = \hat{u}_0(s) e^{-\alpha^2 |s|^2 t}.
\] \\ 
Inverse Quaternion Fourier Integral:

\[
u(q,t) = \int_{\mathbb{H}} \hat{u}_0(s) e^{s \cdot q} e^{-\alpha^2 |s|^2 t} \, ds.
\]

\[
\text{Heat Kernel: } \quad K(q,t) = \frac{1}{(4\pi \alpha^2 t)^2} e^{-\frac{|q|^2}{4\alpha^2 t}}.
\] \\ 
General solutions: 

\begin{align}
\therefore u(q,t) = \int_{\mathbb{H}} K(q - q', t) \, u_0(q') \, dq'.
\end{align} \\ 

\subsection{Hypercomplex Analytic Solutions}
For hyperholomorphic functions, solutions can be expressed as:
\[
u(q,t) = \sum_{n=0}^\infty a_n(t) P_n(q) + \mathcal{O}(\alpha) \hat\alpha. \qed
\]
where \(P_n(q)\) are Fueter polynomials. 

\begin{align}
\therefore f(q) = \sum_{\alpha} c_{\alpha} P_{\alpha}(q) \qed 
\end{align} \\ 
In this expansion, 
\[
f: \Omega \subset \mathbb{H} \to \mathbb{R}
\]
is a regular Fueter function defined in a suitable domain \(\Omega\) in the quaternion algebra, and the Fueter polynomials $P_{\alpha}(q)$ form a basis analogous to the monomials in the Taylor series for complex functions. The coefficients $c_{\alpha}$ are quaternion-valued, encoding the function's behavior in the domain (or pre-image) \(\Omega\), while the series itself maps \(\Omega\) to its image (range) in \(\mathbb{H}\). This representation not only captures the local behavior of \(f(q)\), but also provides a framework for analyzing its global properties by relating the structure of the function to the algebraic and geometric properties of quaternions. \\ \\ 
Finally, we apply our Fourier integral transform for inverse quaternion, denoted by \(\mathcal{F}_Q^{-1}\), to revert our solution back to the real domain. This critical step follows a rigorous treatment of all singular cases arising from the inhomogeneous heat equation, ensuring that the resulting transform accurately captures the solution's behavior. By resolving these singularities, the inverse transform yields a smooth, well-defined function in the real space that reflects the underlying thermal dynamics and adheres to the physical constraints of the problem. \\ \\ 
We express the solution \( u(t,x) \), where \( x \) denotes the measurable spatial variable in the real domain, via the Fourier integral transform inverse quaternion. In particular, if \( R(\xi,t) \) is the transform of real value of a function of quaternion value \( Q(q,t) \), then the inverse transform is given by
\[
u(t,x) = \mathcal{F}_Q^{-1}[R](t,x) = \int_{\mathbb{R}^n} R(\xi,t) \, e^{i \langle \xi, x \rangle} \, d\xi,
\]
where \( e^{i \langle \xi, x \rangle} \) reintroduces the spatial variable \( x \) into the solution. This formulation enables us to recover the original function \( u(t,x) \) from its transform, thus bridging our quaternion-based analysis and the corresponding real-domain representation. Additionally, we have the expansion
\[
\boxed {u(t, \vec x) = \sum_{n=0}^{\infty} a_n(t) P_{\mathbb{R}^{n}}(\vec x) + \frac{1}{O} x\|\mathcal{O}(\alpha)\,\hat{\alpha}\| = c_{o} + c_{n}\sin{\mid (2n-q)^{\gamma}x \mid e^{-\Lambda t}}}  \quad \qed
\]
which, together with the inverse transform, provides a comprehensive framework to solve and analyze our system.


\end{document}